%!TEX root = ../thesis.tex
%*******************************************************************************
%*********************************** First Chapter *****************************
%*******************************************************************************

\chapter{Introduction}  %Title of the First Chapter

\ifpdf
    \graphicspath{{Chapter1/Figs/Raster/}{Chapter1/Figs/PDF/}{Chapter1/Figs/}}
\else
    \graphicspath{{Chapter1/Figs/Vector/}{Chapter1/Figs/}}
\fi


%********************************** %First Section  **************************************
\section{Background} %Section - 1.1 

Graphs are naturally used to model data that consists of entities and the relationships between them, with entities represented as nodes and the relationships between a pair of entities represented as edges. Common applications include modelling communication networks, social networks, and the Web. In such applications, graph data is often available as a massive input stream of edges, so computing exact properties of the graph is often not computationally feasible. Thus, sketches, which summarize general data streams, are often used to approximate queries with a particular guaranteed accuracy.


%********************************** %Second Section  *************************************
\section{Objectives} %Section - 1.2

The objective of this work is to improve gMatrix's \cite{khan} accuracy using the idea of partitioning given data sample \cite{DBLP}. Experiments are to be carried on multiple dat

%********************************** % Third Section  *************************************
\section{Relevant Literatures}  %Section - 1.3 
\label{section1.3}

Previous work \cite{DBLP} has introduced a way to utilize the Count-min sketch \cite{cormode2005improved}, which is commonly used to approximate the frequency of events in a general data stream, to approximate the frequency of edges in a graph stream. Furthermore, it \cite{DBLP} has introduced the gSketch, which is a partitioning scheme that exploits typical local structural properties within real world graph datasets by building multiple localized Count-min sketches \cite{cormode2005improved} to summarize a graph stream. It has been shown that the partitioning scheme is able to improve overall accuracy when tested on several large graph datasets.

A recent work \cite{khan} has introduced the gMatrix, which is a sketch that generalizes the Count-min sketch for graph streams. The gMatrix has been shown to be able to handle various queries involving the structural properties of the underlying graph, which is an advantage over the Count-min sketch \cite{DBLP}.

Applying a partitioning scheme identical to the gSketch \cite{DBLP} into the gMatrix \cite{khan} may improve its overall accuracy. However, there are no previous works that have shown if it really is the case and the extent of the improvement.
